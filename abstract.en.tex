A dissertation or thesis is a document submitted in support of candidature
for a degree or professional qualification presenting the author's research and
findings. In some countries/universities, the word thesis or a cognate is used
as part of a bachelor's or master's course, while dissertation is normally
applied to a doctorate, whilst, in others, the reverse is true.

\fixme{Abstract text goes here.}

\texttt{\textbackslash fixme\{\}} is a command that helps you identify parts of
your thesis that still require some work.
When compiled in the custom \texttt{mydraft} mode, text parts tagged with
fixmes are shown in bold and with fixme tags around them.
When compiled in normal mode, the fixme-tagged text is shown normally (without
special formatting).
The other draft mode command is \texttt{\textbackslash TODO\{\}}
which precedes its argument with a coloured
\TODO{, encloses it in brackets and types it in bold face.}
In normal mode it produces nothing.

The draft mode also causes the ``DRAFT'' text to appear on
the front page, alongside with the document compilation date. The custom
\texttt{mydraft} mode is selected by the \texttt{mydraft} option given for the
package \texttt{aalto-thesis}, near the top of the \texttt{thesis-example.tex}
file.

The thesis example file (\texttt{thesis-example.tex}), all the chapter content
files (\texttt{1introduction.tex} and so on), and the Aalto style file
(\texttt{aalto-thesis.sty}) are commented with explanations on how the Aalto
thesis works. The files also contain some examples on how to customize various
details of the thesis layout, and of course the example text works as an
example in itself. Please read the comments and the example text; that should
get you well on your way!
